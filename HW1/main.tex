\documentclass{article}
\usepackage[utf8]{inputenc}
\usepackage[margin=1.5in]{geometry}
\usepackage{textcomp}
\PassOptionsToPackage{hyphens}{url}\usepackage{hyperref}
\usepackage{adjustbox}
\usepackage{mathcomp}
\usepackage{amsmath,amsfonts,bm}
\usepackage{amsthm, amssymb, mathtools}
\usepackage{mathabx, mathrsfs, dsfont}
\usepackage{graphicx}
\usepackage{tcolorbox}

\usepackage{booktabs, multirow, multicol}
\usepackage{tabularx}

\usepackage{natbib}
\bibliographystyle{plainnat}
\setcitestyle{authoryear}

\newcommand{\diff}{\mathop{}\!\mathrm{d}}
\DeclareMathOperator{\Softmax}{Softmax}

\title{STATS 217 homework 1}
\author{Siping Wang 006405652}
\date{July 2019}

\begin{document}

\maketitle

\section{1.10}

Let $E_1$ denotes the event that the number of rolling ends up with an odd number. \\
Let $E_2$ denotes the event that the number of rolling ends up with an even number. \\
Let $e_1$ denotes the outcome number of the first roll. \\\\
Then, we have: 
\begin{equation}
    P(E_1) + P(E_2) = 1
\end{equation}
and
\begin{equation*}
    P(E_1) = P(e_1 = 3) \times P(E_1|e_1 = 3) + P(e_1 \not= 3) \times P(E_1|e_1 \not= 3).
\end{equation*}
Since that 
\begin{equation*}
    P(E_1|e_1 \not= 3) = P(E_2)
\end{equation*}
and
\begin{equation*}
    P(E_1|e_1 = 3) = 1,
\end{equation*}
we have
\begin{equation}
    P(E_1) = \frac{1}{6} \times 1 + \frac{5}{6} \times P(E_2)
\end{equation}
Solving (1) and (2), we can get that
\begin{equation*}
    \left\{
    \begin{array}{cc}
         P(E_1) = \frac{6}{11}&  \\
         &   \\
         P(E_2) = \frac{5}{11}&  
    \end{array}
    \right.
\end{equation*}
So the probability that an even number of rolls is needed is $\frac{5}{11}$.

\section{1.17}

\begin{align*}
    E(X|X > 2) &= \sum\limits_x x P(X=x|X>2)  \\
    &= \sum\limits_x \frac{P(X = x, X > 2)}{P(X > 2)} \\
    &= \frac{\sum\limits_{x = 3}^\infty x P(X = x)}{\sum\limits_{x = 3}^\infty P(X = x)} \\
    &= \frac{E(X) - P(X = 1) - 2 P(X = 2)}{1 - P(X = 0) - P(X = 1) - P(X = 2)}.
\end{align*}
Since that
\begin{align*}
    E(X) &= \lambda = 3, \\
    P(x = 0) &= \frac{e^{-3}}{0!} = e^{-3}, \\
    P(x = 1) &= \frac{3 \times e^{-3}}{1!} = 3 \times e^{-3}, \\
    P(x = 2) &= \frac{3^2 \times e^{-3}}{2!} = \frac{9 \times e^{-3}}{2},
\end{align*}
we have
\begin{align*}
    E(X|X > 2) &= \frac{3 - 12 e^{-3}}{1 - \frac{17}{2} e^{-3}} \\
    &\approx 4.16525.
\end{align*}

\section{1.22}
\subsection{(b)}
\begin{align*}
    E(Y|X) &= \int_0^x y f_{Y|X}(y|X = x) dy \\
    &= \int_0^x y \frac{f_{xy}(x, y)}{f_X(x)} dy \\
    &= \int_0^x y \frac{f_{xy}(x, y)}{\int_0^x f_{xy}(x, y) dy} dy.
\end{align*}
Since that $(X, Y)$ is uniformly distributed on the triangle, 
\begin{equation*}
    f_{xy}(x, y) = \frac{1}{S_{\Delta}} = \frac{1}{\frac{1}{2}} = 2.
\end{equation*}
Then
\begin{align*}
    E(Y|X) &= \int_0^x y \frac{2}{\int_0^x 2 dy} dy \\
    &= \int_0^x y \frac{2}{2x} dy \\
    &= \frac{1}{x}\times \frac{1}{2} x^2 \\
    &= \frac{x}{2}.
\end{align*}

\subsection{(c)}
\begin{align*}
    E(Y|X) &= \int_{-\sqrt{1-x^2}}^{\sqrt{1-x^2}} y f_{Y|X}(y|X = x) dy \\
    &= \int_{-\sqrt{1-x^2}}^{\sqrt{1-x^2}} y \frac{f_{xy}(x, y)}{f_X(x)} dy \\
    &= \int_{-\sqrt{1-x^2}}^{\sqrt{1-x^2}} y \frac{f_{xy}(x, y)}{\int_{-\sqrt{1-x^2}}^{\sqrt{1-x^2}} f_{xy}(x, y) dy} dy.
\end{align*}
Since that $(X, Y)$ is uniformly distributed on the disc, 
\begin{equation*}
    f_{xy}(x, y) = \frac{1}{S_{O}} = \frac{1}{\pi}.
\end{equation*}
Then
\begin{align*}
    E(Y|X) &= \int_{-\sqrt{1-x^2}}^{\sqrt{1-x^2}} y \frac{\frac{1}{\pi}}{\int_{-\sqrt{1-x^2}}^{\sqrt{1-x^2}} \frac{1}{\pi} dy} dy \\
    &= \int_{-\sqrt{1-x^2}}^{\sqrt{1-x^2}} y \frac{1}{2\sqrt{1-x^2}} dy \\
    &= \frac{1}{2\sqrt{1-x^2}} \times 0 \\
    &= 0.
\end{align*}

\section{1.26}
\begin{align*}
    P(Y<2) &= \int_0^2 f_Y(y) dy \\
    &= \int_0^2 dy \int_y^{\infty} f_{xy}(x, y) dx \\
    &= \int_0^2 dy \int_y^{\infty} f_X(x) f_{Y|X}(Y|X = x) dx \\
    &= \int_0^2 dy \int_y^{\infty} x e^{-x} \frac{1}{x} dx \\
    &= \int_0^2 dy \int_y^{\infty} e^{-x} dx \\
    &= \int_0^2 \frac{1}{e^y} dy \\
    &= 1 - \frac{1}{e^2} \approx 0.865.
\end{align*}

\section{1.31}
Let $e_1$ denotes the result of the first trail. That is, $e_1 = 1$ means success, otherwise $e_1 = 0$. \\
Since that 
\begin{equation*}
    X \sim Ge(p),
\end{equation*}
we have
\begin{align}
    E(X) &= P(e_1 = 1) \times E(X|e_1 = 1) + P(e_1 = 0) \times E(X|e_1 = 0) \\
    &= p E(X|e_1 = 1) + (1 - p) E(X|e_1 = 0) \\
    &= p \times 1 + (1 - p) (E(X) + 1).
\end{align}
(5) solves that
\begin{equation*}
    E(X) = \frac{1}{p}.
\end{equation*}
Since that
\begin{align*}
    E(Var(X|e_1)) &= (1 - p) Var(X|e_1 = 0) + p Var(X|e_1 = 1) \\
    &= (1 - p) Var(X), \\
\end{align*}
and
\begin{align*}
    Var(E(X|e_1)) &= E((E(X|e_1))^2) - (E(E(X|e_1)))^2 \\
    &= E((E(X|e_1))^2) - \frac{1}{p^2} \\
    &= (1 - p) E((E(X|e_1 = 0))^2) + p E((E(X|e_1 = 1))^2) - \frac{1}{p^2} \\
    &= (1 - p)E((E(X) + 1)^2) + p - \frac{1}{p^2} \\
    &= (1 - p) (\frac{1}{p} + 1)^2 + p - \frac{1}{p^2} \\
    &= \frac{1 - p}{p},
\end{align*}
we have
\begin{align}
    Var(X) &= E(Var(X|e_1)) + Var(E(X|e_1)) \\
    &= (1 - p) Var(X) + \frac{1 - p}{p}. 
\end{align}
(7) solves that
\begin{equation*}
    Var(X) = \frac{1 - p}{p^2}.
\end{equation*}

\end{document}
